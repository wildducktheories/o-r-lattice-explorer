\documentclass[12pt]{article}
\usepackage{amsmath, amssymb, amsfonts}
\usepackage{hyperref}
\usepackage{geometry}
\geometry{margin=1in}

\title{Affine Block Structure in Collatz Sequences}
\author{Jon Seymour}
\date{\today}

\begin{document}

\maketitle

\begin{abstract}
We present a simplified framework for analyzing Collatz sequences through affine block structures. Each odd block is characterized by three parameters $(\alpha, \beta, \rho)$ and a scaling parameter $t$ that enumerates instances of the block. By restricting attention to odd blocks whose successors are also odd, the framework achieves a minimal parameterization. Odd blocks are instances of Steiner circuits as defined by Steiner (1977). The framework defines two affine functions: $x(B,t)$ mapping $t$ to odd integers, and $x^{\rightarrow}(B,t)$ giving the odd integer at the start of the next Steiner circuit. This approach focuses exclusively on lattice-wide affine relationships without attempting to model internal block dynamics.
\end{abstract}

\section*{Revisions}

\begin{itemize}
    \item \textbf{2025-01-20}: Simplified to consider only odd blocks (removed $\nu$ parameter). Restricted to blocks where $x^{\rightarrow}$ is also odd, with $\beta = v_2(3^\alpha \rho - 1)$. Changed parameterization from $(\alpha, \nu, \rho, \kappa)$ to $(\alpha, \beta, \rho)$. Adopted $x^{\rightarrow}$ notation for the successor (first odd of next Steiner circuit). Added reference to Steiner circuits. Added $t=1$ example.
\end{itemize}

\section{Introduction}

Collatz sequences exhibit structure that can be analyzed through \emph{blocks}---contiguous subsequences with predictable parity patterns. This work presents a minimal parameterization of such blocks that captures their essential affine properties.

Following Steiner \cite{steiner1977}, we observe that odd blocks---those starting from odd integers---form what he termed ``circuits'' in the Collatz graph. By restricting attention to odd blocks where $x^{\rightarrow}$ (the first odd of the next circuit) is also odd, we achieve a particularly clean representation.

The key insight is to represent odd blocks using the fundamental identity:
\[
x = 2^\alpha \bar{\rho} - 1, \quad \text{where } \bar{\rho} = \rho + t \cdot 2^{\beta + 1}
\]

This identity leads to a clean 3-parameter representation $(\alpha, \beta, \rho)$ that avoids the complications of tracking internal 3-adic structure or leading even steps.

\section{Block Parameters}

An odd block $B$ is defined by three parameters and a scaling parameter:
\[
B = (\alpha, \beta, \rho), \quad t \ge 0
\]

where:
\begin{itemize}
    \item $\alpha \ge 1$ is the 2-adic valuation $v_2(x + 1)$
    \item $\rho \ge 1$ is an odd integer parameter
    \item $\beta = v_2(3^\alpha \rho - 1)$, determining the block's even tail
    \item $t \ge 0$ is the scaling parameter enumerating block instances
\end{itemize}

Let $\bar{\rho} = \rho + t \cdot 2^{\beta + 1}$. Since $3^\alpha \bar{\rho} - 1 = (3^\alpha \rho - 1) + 3^\alpha t \cdot 2^{\beta+1}$, we have $v_2(3^\alpha \bar{\rho} - 1) = \beta$ for all $t \ge 0$.

The block length (total number of even steps) is $\kappa = \alpha + \beta$. By restricting to blocks where $x^{\rightarrow}$ is also odd, the value $(3^\alpha \bar{\rho} - 1)/2^\beta$ is guaranteed to be an odd integer.

\section{Affine Functions}

The block parameters $(\alpha, \beta, \rho)$ define two affine functions of $t$, with $\bar{\rho} = \rho + t \cdot 2^{\beta + 1}$.

\subsection{The x-Function}

\[
x(B,t) = 2^\alpha \bar{\rho} - 1 = 2^\alpha (\rho + t \cdot 2^{\beta + 1}) - 1
\]

This is an affine function with slope $m_x = 2^{\alpha + \beta + 1}$ and intercept $c_x = 2^\alpha \rho - 1$.

\subsection{The Successor Function}

We write $x^{\rightarrow}$ to denote the odd integer at the start of the next Steiner circuit---that is, the first odd value reached after completing the current block's sequence of Collatz operations.

\[
x^{\rightarrow}(B,t) = \frac{3^\alpha \bar{\rho} - 1}{2^\beta} = \frac{3^\alpha \cdot (\rho + t \cdot 2^{\beta + 1}) - 1}{2^\beta}
\]

Expanding:
\[
x^{\rightarrow}(B,t) = \frac{3^\alpha \rho - 1}{2^\beta} + 2 \cdot 3^\alpha \cdot t
\]

This is an affine function with slope $m_{x^{\rightarrow}} = 2 \cdot 3^\alpha$ and intercept $c_{x^{\rightarrow}} = (3^\alpha \rho - 1)/2^\beta$.

Since $v_2(3^\alpha \bar{\rho} - 1) = \beta$ for all $t$, the successor $x^{\rightarrow}(B,t)$ is always an odd integer.

\section{Computing Block Parameters}

Given an odd integer $x$, we compute its block parameters as follows:

\begin{enumerate}
    \item Compute $\alpha = v_2(x + 1)$
    \item Compute $\bar{\rho} = (x + 1) / 2^\alpha$
    \item Compute $\rho = \bar{\rho} \bmod 2^{\beta + 1}$, where $\beta = v_2(3^\alpha \rho - 1)$
    \item Compute the scaling parameter:
    \[
    t = \left\lfloor \frac{\bar{\rho} - \rho}{2^{\beta + 1}} \right\rfloor
    \]
\end{enumerate}

\section{Verification}

The computed parameters can be verified by checking:
\[
x = 2^\alpha \bar{\rho} - 1 = 2^\alpha (\rho + t \cdot 2^{\beta + 1}) - 1
\]

This should exactly equal the original $x$ value.

\section{Examples}

\subsection{Example: $x = 35$ (t = 0)}

For $x = 35$:
\begin{itemize}
    \item $\alpha = v_2(36) = 2$
    \item $\bar{\rho} = 36/4 = 9$
    \item $\beta = v_2(3^2 \cdot 9 - 1) = v_2(80) = 4$
    \item $\rho = 9 \bmod 32 = 9$
    \item $t = \lfloor(9-9)/32\rfloor = 0$
\end{itemize}

The block parameters are: $B = (\alpha=2, \beta=4, \rho=9)$, $t=0$

The affine functions are:
\begin{align*}
x(t) &= 2^2(9 + t \cdot 2^5) - 1 = 4(9 + 32t) - 1 = 128t + 35 \\
x^{\rightarrow}(t) &= \frac{3^2 \cdot 9 - 1}{2^4} + 2 \cdot 3^2 \cdot t = \frac{80}{16} + 18t = 5 + 18t
\end{align*}

For $t = 0$: $x(0) = 35$ and $x^{\rightarrow}(0) = 5$.

Indeed, starting from $x = 35$, the Collatz map gives: $35 \to 106 \to 53 \to 160 \to 80 \to 40 \to 20 \to 10 \to 5$, confirming that $x^{\rightarrow} = 5$.

\subsection{Example: $x = 163$ (t = 1)}

Using the same block $B = (\alpha=2, \beta=4, \rho=9)$ with $t = 1$:
\begin{itemize}
    \item $x(1) = 128 \cdot 1 + 35 = 163$
    \item $x^{\rightarrow}(1) = 5 + 18 \cdot 1 = 23$
\end{itemize}

Verification: For $x = 163$:
\begin{itemize}
    \item $\alpha = v_2(164) = 2$
    \item $\bar{\rho} = 164/4 = 41$
    \item $\beta = v_2(3^2 \cdot 41 - 1) = v_2(368) = 4$
    \item $\rho = 41 \bmod 32 = 9$
    \item $t = \lfloor(41-9)/32\rfloor = 1$
\end{itemize}

The Collatz sequence: $163 \to 490 \to 245 \to 736 \to 368 \to 184 \to 92 \to 46 \to 23$, confirming that $x^{\rightarrow} = 23$.

\section{Significance}

This framework provides several insights:

\begin{itemize}
    \item \textbf{Affine structure}: Blocks naturally organize into affine families, revealing geometric patterns in Collatz sequences
    \item \textbf{Minimal parameterization}: Using only 3 parameters $(\alpha, \beta, \rho)$ plus $t$, we capture the essential structure without internal dynamics
    \item \textbf{Steiner circuits}: Odd blocks correspond to Steiner's circuits, providing a connection to established terminology in Collatz research
    \item \textbf{Lattice-wide relationships}: The $x^{\rightarrow}$ function connects successive Steiner circuits across trajectories
    \item \textbf{Computational efficiency}: Block parameters can be computed directly from odd $x$ without iterating the sequence
\end{itemize}

\section{Scope and Limitations}

This framework intentionally focuses on \emph{lattice-wide affine relationships} between odd blocks as atomic units. It does not attempt to model:

\begin{itemize}
    \item Even starting values
    \item Internal evolution of individual blocks through the Collatz map
    \item 3-adic structure within blocks (powers of 3 in intermediate values)
    \item Step-by-step parity patterns within blocks
\end{itemize}

By restricting to odd blocks where $x^{\rightarrow}$ is also odd, we obtain a particularly clean framework. Even starting values and more general block structures can be treated as extensions if needed.

\section{Conclusion}

The affine block framework provides a clean, minimal approach to understanding Collatz sequences through their odd block structure. By focusing exclusively on affine relationships between odd blocks---instances of Steiner circuits---and avoiding the complexities of even values and internal dynamics, this approach reveals the geometric organization of Collatz sequences while maintaining mathematical rigor and computational tractability.

\begin{thebibliography}{9}
\bibitem{steiner1977}
R. P. Steiner,
``A theorem on the Syracuse problem,''
\textit{Proceedings of the 7th Manitoba Conference on Numerical Mathematics and Computing},
pp. 553--559, 1977.
\end{thebibliography}

\end{document}
